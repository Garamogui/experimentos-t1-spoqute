\chapter{Introdução}

A química é uma ciência fundamental que estuda a composição, estrutura, propriedades e transformações da matéria. Dentre os diversos tópicos de pesquisa na área, a condutividade elétrica e a solubilidade de substâncias em diferentes solventes são de particular interesse, uma vez que influenciam diretamente em diversas aplicações práticas, como na indústria farmacêutica, na produção de alimentos e na formulação de produtos químicos.

A condutividade elétrica de uma solução é determinada pela presença de íons, que são portadores de carga elétrica. Soluções iônicas, como o cloreto de sódio (NaCl), geralmente apresentam alta condutividade devido à dissociação completa dos íons em solução. Em contrapartida, soluções moleculares, como a sacarose, não conduzem eletricidade, pois não se dissociam em íons.

Além disso, a solubilidade de compostos em diferentes solventes é um aspecto crucial na química, pois determina a viabilidade de reações químicas e a formação de soluções. A água, conhecida como "solvente universal", é capaz de dissolver uma ampla gama de substâncias, enquanto solventes orgânicos, como o hexano, apresentam características diferentes que afetam a solubilidade de compostos.

Este trabalho tem como objetivo testar a condutividade de diferentes tipos de soluções iônicas e moleculares, bem como investigar a solubilidade de diversos solutos em água e em solventes orgânicos. Os resultados obtidos permitirão uma melhor compreensão dos princípios que regem a condutividade e a solubilidade, contribuindo para o conhecimento na área da química.
