\chapter{Fundamentação Teórica}


\section{Condutividade de compostos e soluções iônicas e moleculares.}\label{sub:Condutividade de compostos e soluções iônicas e moleculares.} % (fold)
% section Condutividade de composto (end)
A condutividade elétrica de um material está ligado com a capacidade que o mesmo possui de conduzir corrente elétrica.\cite{lab-quimica-condutividade-eletrica} Podem apresentar diferentes níveis de condução, sendo classificados como condutores, semicondutores ou isolantes.

A corrente elétrica é definida como um movimento ordenado de partículas eletricamente carregadas, que circulam em um material condutor que possua uma diferença de potencial. Então, para que se consiga acender uma lâmpada, na qual seus terminais estejam submergidos numa solução, deve haver uma diferença de potencial entre estes terminais. 

As partículas com carga são conhecidas como eletrólitos, que são capazes de se dissociar em cátions e ânions quando estão numa constante dielétrica. Os íons se movem simultaneamente na solução, se repelindo e assim estabelecendo uma corrente elétrica.

A condutividade de uma solução está ligada com sua concentração. Se faz necessária a presença de íons para a condução, por exemplo: a água em seu estado puro conduz muito pouca corrente elétrica, porém se adicionarmos eletólitos (pode ser $NaCl$) ela se torna mais condutora. Estudos mostram que a condutividade não aumenta proporcionalmente à quantidade de eletrólitos adicionados na substância, devido as interações entre os íons na solução.

\subsection{Solubilidade de Compostos}\label{sub:Condutividade de compostos e soluções iônicas e moleculares.} % (fold)

A solubilidade de uma substância pode ser observada quando há a interação entre uma substância que será solubilizada, conhecida como soluto, e uma substância que dissolve, conhecida como solvente \cite{solubilidade-substancias-organicas}. Pode ser definida dependendo da quantidade de soluto que é dissolvido. 

A solubilidade está ligada a fatores de estrutura molecular. Os compostos que não são polares são soluveis em compostos não polares, já os polares serão solúveis em compostos polares. Logo, a solubilidade está atrelada a atração intramoleculares.

De forma resumida, a atração entre particulas do soluto e do solvente devem ser fortes o suficiente para compensar o rompimento das ligações dos próprios componentes da interação, assim possibilitando a mistura de ambos os compostos. Ao analisar as forças quando misturados, a variação de entropia devem ser suficiente para que a entalpia seja negativa, zero ou fracamente positiva. Caso isso não ocorra, a variação de entropia não será forte suficiente para realizar uma dissolução espontânea.
