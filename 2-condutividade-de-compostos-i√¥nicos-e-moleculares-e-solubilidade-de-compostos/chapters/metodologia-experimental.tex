\chapter{Metodologia Experimental}

\section{Materiais e Reagentes}\label{sec:Materiais e Reagentes} % (fold)

% section Materiais e Reagentes (end)
\begin{itemize}
 \item Béquer de 100$ml$;
 \item Tubos de ensaio;
 \item Estante para tubos de ensaio;
 \item Sistema com lâmpada para teste de condutividade;
 \item Balança analítica;
 \item Bastaão de vidro ;
 \item Cloreto de sódio ($NaCl$);
 \item Açúcar (sacarosa);
 \item Solução de ácido clorídrico $(HCl)$ de 0,1\textbf{$mol/L$};
 \item Solução de Hidróxido de Sódio $(NaOH)$ de 0,1\textbf{$mol/L$};
 \item Etanol;
 \item Óleo de soja;
 \item Hexano;
 \item Pisseta com água destilada.
\end{itemize}

\section{Procedimento Experimental}\label{sec:Procedimento Experimental} % (fold)

\subsection{Condutividade de compostos e soluções iônicas e moleculares.}\label{sub:Condutividade de compostos e soluções iônicas e moleculares.} % (fold)

- \textbf{Ligar na tomada elétrica a lâmpada}, tomando cuidado para que os fios metálicos não encostem um com o outro (e o mesmo não tocar com as mãos)e \textbf{verifique a condutividade}, colocando os eletrodos imersos, nas seguintes soluções:

\begin{enumerate}
 \item 50 ml de água destilada;
 \item 50 ml de água destilada com 1,0g de sal $NaCl$;
 \item 50 ml de água destilada com 1,0g de açúcar;
 \item 30 ml de solução aquosa de ácido clorídrico $HCl_{aq}$ com $0,1 mol/L$;
 \item 30 ml de solução aquosa de hidróxido de sódio $NaOH$ ($0,1 mol/L$);
 \item 30 ml de solução aquosa de ácido acético (vinagre);
 \item 30 ml de ácool etílico (etanol);
 \item 30 ml de óleo de soja.
\end{enumerate}

\subsection{Solubilidade de Compostos}\label{sub:Condutividade de compostos e soluções iônicas e moleculares.} % (fold)

\begin{enumerate}
 \item Colocar 9 tubos de ensaio em uma estante para tubos e etiquetar com cada uma das condições dos testes de solubilidade.
 \item Em cada tubo de ensaio, adicionar a mesma quantidade de soluto a ser testado, considerando as substâncias indicadas na tabela abaixo. Utilizar cerca de 50mg ou menos se o soluto for sólido (a massa deve ser suficiente para formar um pequeno punhado de sólido centralizado no fundo do tubo de ensaio) e em torno de 1 mL de soluto, se o mesmo for líquido.
 \item Para cada soluto (1,2,3,4 ou 5), testar a solubilidade nos solventes denominados pelas letras A e B, colocando inicialmente quantidades semelhantes de solvente, em torno de 2mL.
 \item Se necessário, com a finalidade de se verificar a solubilidade do soluto, adicionar mais volumes de solvente, até no máximo 3mL de líquido por tubo. Adicione a mesma quantidade de solvente nos outros tubos que contêm o mesmo soluto, para efeitos comparativos.
\end{enumerate}

