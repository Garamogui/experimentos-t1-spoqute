
\chapter{Conclusões}
Parte I: Reatividade entre metais e íons metálicos
\begin{itemize}
 \item Fragmentos de cobre em $Mg^{2+}{(aq)}$ e $Zn^{2+}{(aq)}$: Não deve haver grande reatividade. O cobre é menos reativo que o magnésio e o zinco, então deve permanecer inerte.
 \item Fragmentos de magnésio em $Cu^{2+}{(aq)}$ e $Zn^{2+}{(aq)}$: Deve haver reatividade visível com $Cu^{2+}{(aq)}$, porque o magnésio é mais reativo. Menos reatividade com $Zn^{2+}{(aq)}$.
 \item Fragmentos de alumínio em $Cu^{2+}{(aq)}$, $Mg^{2+}{(aq)}$, $Zn^{2+}{(aq)}$: A reação mais visível será com $Cu^{2+}{(aq)}$, pois o alumínio é bem reativo.
\end{itemize}


Parte II: Reatividade dos metais com ácido clorídrico ($HCl$)
\begin{itemize}
 \item Fragmentos de cobre com $HCl$: Pouca ou nenhuma reatividade. O cobre não reage facilmente com ácidos diluídos.
 \item Fragmentos de magnésio com $HCl$: Reatividade alta, com formação de bolhas de hidrogênio ($H_2$).
 \item Fragmentos de zinco com $HCl$: Reatividade moderada a alta, com liberação de gás hidrogênio.
 \item Fragmentos de alumínio com $HCl$: Reatividade notável, com formação de gás hidrogênio.
\end{itemize}

