\chapter{Fundamentação Teórica}
A reatividade dos metais é um aspecto central na química, influenciando tanto processos naturais quanto aplicações industriais. A série de reatividade dos metais, também conhecida como série eletroquímica, classifica os metais em ordem de sua tendência a perder elétrons (ser oxidados). Esse conceito é essencial para entender como diferentes metais interagem com ácidos e com outros íons metálicos.

Metais como o magnésio ($Mg$) e o alumínio ($Al$) estão no topo da série de reatividade, indicando uma alta propensão a oxidar. Quando esses metais são introduzidos em uma solução de ácido clorídrico ($HCl$), eles reagem vigorosamente, formando gás hidrogênio $H_{2}$ e um sal metálico.

Metais menos reativos como o cobre (Cu) mostram pouca ou nenhuma reação com ácidos diluídos. Este comportamento é devido ao menor potencial de ionização do cobre, que o torna mais estável e menos propenso a perder elétrons.

Além disso, as reações de deslocamento entre metais e íons metálicos em solução dependem dos potenciais de redução dos metais envolvidos. Um metal com um potencial de redução mais baixo tende a ser oxidado quando colocado em uma solução contendo íons metálicos de um metal com um potencial de redução mais alto. 
\cite{brown-electronegativity}

No contexto deste experimento, a observação de mudanças físicas como variações de temperatura, formação de bolhas, liberação de gás e mudanças de cor fornecerá uma avaliação qualitativa da reatividade dos metais. Esses dados permitirão uma compreensão mais detalhada da posição dos metais na série de reatividade e suas características de oxirredução.


