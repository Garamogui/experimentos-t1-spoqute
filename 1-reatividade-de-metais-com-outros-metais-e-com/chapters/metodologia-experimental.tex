\chapter{Metodologia Experimental}
\begin{itemize}
 \section{Materiais e Reagentes}

  \item 1 béquer de 100 mL;
  \item 1 tubo de ensaio;
  \item Estante para tubos de ensaio;
  \item Bastão de video;
  \item Solução de ácido clorídrico 1,0 mol/L;
  \item Cobre em aparas;
  \item Solução de sulfato de Cobre II ($CuSO_4$);
  \item Magnésio em aparas;
  \item Solução de sulfato de magnésio ($MgSO_4$);
  \item Zinco em aparas;
  \item Solução de sulfato de zinco ($ZnSO_4$);
  \item Solução de Sulfato de alumínio ($Al_2(SO_4)_3$);
  \item Alumínio em aparas;

 \section{Procedimento Experimental}
 \subsection{Parte I}
 \begin{enumerate}
  \item Adicione ceca de 2 mL das soluções de íons metálicos: $Cu^{2+}_{(aq)}$, $Mg^{2+}_{(aq)},$ $Al^{2+}_{(aq)}$
  \item Adicione fragmentos dos metais nas soluções de íons metálicos não equivalentes ao seu próprio íon (isto é: adicione raspas de cobre nas soluções de $Mg^{2+}_{(aq)}$ $Zn^{2+}_{(aq)}$, mas não a de $Cu^{2+}_{(aq)}$), e assim sucessivamente; )

  \item Observe a reatividade de cada par metal/íon metálico e anote: aumento de temperatura, formação de bolhas, liberação de gás, corrosão, etc. 

 \end{enumerate}

 \subsection{Parte II}
 \begin{enumerate}
  \item Adicione cerca de 2 mL da solução de $HCL$ em cinco tubos de ensaio;
  \item Adicione fragmentos dos metais cobre, magnésio, zinco e alumínio em cada tubo de ensaio com a solução de HCL;
  \item Observe a reatividade de cada metal em meio ácido e anote: aumento de temperatura, formação de bolhas, liberação de gás, corrosão, etc.

 \end{enumerate}

\end{itemize}
