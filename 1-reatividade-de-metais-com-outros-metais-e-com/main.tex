\documentclass[ 12pt, % tamanho da fonte
 openright, % capítulos começam em página ímpar (insere página vazia caso preciso)
 oneside, % para impressão em verso e anverso. Oposto a oneside
 a4paper, % tamanho do papel. 
 english, % idioma adicional para hifenização
 brazil % o último idioma é o principal do documento
 ]{abntex2}

\usepackage{booktabs}
\usepackage{times}
\usepackage{multirow}
\usepackage[utf8]{inputenc} 
\usepackage{graphicx} % Required for inserting images
\usepackage{array}
\usepackage{csquotes}
\usepackage[bibstyle=abnt, citestyle=abnt-numeric]{biblatex}
\usepackage{capa}
\addbibresource{$HOME/Documents/latex/referencias.bib}


\newcolumntype{P}[1]{>{\centering\arraybackslash}p{#1}}

\autor{ \textbf{Estudantes:} Gabriel Arom Gonçalves Vianna de Moura e Silva,\\
 Luigi Gava Costa,\\
 Pedro Henrique Batista,\\
 Rafael Fernandes Barnabé
}
\disciplina{Química Experimental }
\data{2024}
\tipotrabalho{Relatório}
\local{São Paulo}
\titulo{Reatividade de metais com outros metais e com solução ácida.} % introduz título do trabalho
\orientador{Prof. Lucia Collet, \\ Prof. Caio Souza}
\instituicao{Instituto Federal de Educação, Ciência e Tecnologia de São Paulo}

%%%%%%%%%%%%%% Início do documento %%%%%%%%%%%%%%

\begin{document}

\inserircapa

\tableofcontents
\chapter{Introdução}

\chapter{Fundamentação Teórica}


\section{Condutividade de compostos e soluções iônicas e moleculares.}\label{sub:Condutividade de compostos e soluções iônicas e moleculares.} % (fold)
% section Condutividade de composto (end)
A condutividade elétrica de um material está ligado com a capacidade que o mesmo possui de conduzir corrente elétrica.\cite{lab-quimica-condutividade-eletrica} Podem apresentar diferentes níveis de condução, sendo classificados como condutores, semicondutores ou isolantes.

A corrente elétrica é definida como um movimento ordenado de partículas eletricamente carregadas, que circulam em um material condutor que possua uma diferença de potencial. Então, para que se consiga acender uma lâmpada, na qual seus terminais estejam submergidos numa solução, deve haver uma diferença de potencial entre estes terminais. 

As partículas com carga são conhecidas como eletrólitos, que são capazes de se dissociar em cátions e ânions quando estão numa constante dielétrica. Os íons se movem simultaneamente na solução, se repelindo e assim estabelecendo uma corrente elétrica. \cite{brown} 

A condutividade de uma solução está ligada com sua concentração. Se faz necessária a presença de íons para a condução, por exemplo: a água em seu estado puro conduz muito pouca corrente elétrica, porém se adicionarmos eletólitos (pode ser $NaCl$) ela se torna mais condutora. Estudos mostram que a condutividade não aumenta proporcionalmente à quantidade de eletrólitos adicionados na substância, devido as interações entre os íons na solução.

\subsection{Solubilidade de Compostos}\label{sub:Condutividade de compostos e soluções iônicas e moleculares.} % (fold)

A solubilidade de uma substância pode ser observada quando há a interação entre uma substância que será solubilizada, conhecida como soluto, e uma substância que dissolve, conhecida como solvente \cite{solubilidade-substancias-organicas}. Pode ser definida dependendo da quantidade de soluto que é dissolvido. 

A solubilidade está ligada a fatores de estrutura molecular. Os compostos que não são polares são soluveis em compostos não polares, já os polares serão solúveis em compostos polares. Logo, a solubilidade está atrelada a atração intramoleculares.

De forma resumida, a atração entre particulas do soluto e do solvente devem ser fortes o suficiente para compensar o rompimento das ligações dos próprios componentes da interação, assim possibilitando a mistura de ambos os compostos. Ao analisar as forças quando misturados, a variação de entropia devem ser suficiente para que a entalpia seja negativa, zero ou fracamente positiva. Caso isso não ocorra, a variação de entropia não será forte suficiente para realizar uma dissolução espontânea.

\chapter{Metodologia Experimental}

\section{Materiais e Reagentes}\label{sec:Materiais e Reagentes} % (fold)

% section Materiais e Reagentes (end)
\begin{itemize}
 \item Béquer de 100$ml$;
 \item Tubos de ensaio;
 \item Estante para tubos de ensaio;
 \item Sistema com lâmpada para teste de condutividade;
 \item Balança analítica;
 \item Bastaão de vidro ;
 \item Cloreto de sódio ($NaCl$);
 \item Açúcar (sacarosa);
 \item Solução de ácido clorídrico $(HCl)$ de 0,1\textbf{$mol/L$};
 \item Solução de Hidróxido de Sódio $(NaOH)$ de 0,1\textbf{$mol/L$};
 \item Etanol;
 \item Óleo de soja;
 \item Hexano;
 \item Pisseta com água destilada.
\end{itemize}

\section{Procedimento Experimental}\label{sec:Procedimento Experimental} % (fold)

\subsection{Condutividade de compostos e soluções iônicas e moleculares.}\label{sub:Condutividade de compostos e soluções iônicas e moleculares.} % (fold)

- \textbf{Ligar na tomada elétrica a lâmpada}, tomando cuidado para que os fios metálicos não encostem um com o outro (e o mesmo não tocar com as mãos)e \textbf{verifique a condutividade}, colocando os eletrodos imersos, nas seguintes soluções:

\begin{enumerate}
 \item 50 ml de água destilada;
 \item 50 ml de água destilada com 1,0g de sal $NaCl$;
 \item 50 ml de água destilada com 1,0g de açúcar;
 \item 30 ml de solução aquosa de ácido clorídrico $HCl_{aq}$ com $0,1 mol/L$;
 \item 30 ml de solução aquosa de hidróxido de sódio $NaOH$ ($0,1 mol/L$);
 \item 30 ml de solução aquosa de ácido acético (vinagre);
 \item 30 ml de ácool etílico (etanol);
 \item 30 ml de óleo de soja.
\end{enumerate}

\subsection{Solubilidade de Compostos}\label{sub:Condutividade de compostos e soluções iônicas e moleculares.} % (fold)

\begin{enumerate}
 \item Colocar 9 tubos de ensaio em uma estante para tubos e etiquetar com cada uma das condições dos testes de solubilidade.
 \item Em cada tubo de ensaio, adicionar a mesma quantidade de soluto a ser testado, considerando as substâncias indicadas na tabela abaixo. Utilizar cerca de 50mg ou menos se o soluto for sólido (a massa deve ser suficiente para formar um pequeno punhado de sólido centralizado no fundo do tubo de ensaio) e em torno de 1 mL de soluto, se o mesmo for líquido.
 \item Para cada soluto (1,2,3,4 ou 5), testar a solubilidade nos solventes denominados pelas letras A e B, colocando inicialmente quantidades semelhantes de solvente, em torno de 2mL.
 \item Se necessário, com a finalidade de se verificar a solubilidade do soluto, adicionar mais volumes de solvente, até no máximo 3mL de líquido por tubo. Adicione a mesma quantidade de solvente nos outros tubos que contêm o mesmo soluto, para efeitos comparativos.
\end{enumerate}


\chapter{Resultados e Discussão}

\section{Parte I}

\subsection{Tabela de Resultados por observação:}

\begin{center}
 \begin{tabular}{|c| P{3cm} | P{2.5cm} | P{2.5cm} | P{2.0cm} |}
  \hline
  \multirow{2}{3em}{\textbf{Metal}} & \multicolumn{4}{c|}{\textbf{Solução de Íons Metálicos}} \\
  \cmidrule(rl){2-5} 
  \centering
 & $Cu^{2+}_{aq}$ & $Mg^{2+}_{aq}$ & $Zn^{2+}_{aq}$ & $Al^{2+}_{aq}$ \\
  \hline
 Cobre &  \multicolumn{4}{c|}{Não houveram reações significativas} \\
  \hline
 Magnésio & Oxigenou & Oxidou / Temperatura & Oxidou / Boiou & Oxidou  \\ 
  \hline
 Zinco & Cor mudou & Cor Mudou & Cor Mudou  & Oxigenou \\ 
  \hline
 Alumínio & \multicolumn{4}{c|}{Não houveram reações significativas} \\
  \cmidrule(rl){2-5} 
 (Ácido adicionado) &  \parbox{3cm}{- Formação de Sólido Castanho \\- Alumínio Dissolveu \\ - Oxidou} & n/a & \parbox{2.5cm}{\begin{itemize}
  \item Mais Refletivo
  \item Corrosão nas bordas.
 \end{itemize}} & n/a \\ 

  \hline
 
 \end{tabular}
\end{center}


\section{Parte II}
\subsection{Tabela de Resultados por observação:}

\begin{center}
 \begin{tabular}{|c|c|}
  \hline
  \textbf{Metal} & \textbf{Solução de HCL} \\
  \hline
  Cobre & Não houveram reações significativas \\
  \hline
  Magnésio & Aumento de temperatura (qualitativo) e com aparente oxidação. \\
  \hline
  Zinco & Sinais de Corrosão \\
  \hline
  Alumínio &  Não houveram reações significativas  \\
  \hline
 \end{tabular}
\end{center}


\chapter{Conclusões}

\printbibliography
\end{document}
