\chapter{Conclusões}

Pode se concluir, que ao investigar a condutividade elétrica de inúmeras soluções e avaliar a solubilidade nos solventes, os resultados nos trazem a confirmação de que a condutividade elétrica depende de íons na solução: compostos iônicos, como NaCl e HCl, apresentaram alta condutividade ao se dissociarem em íons, enquanto as soluções moleculares, como a sacarose, não conduzem eletricidade devido sua ausência de íons.

Em relação as praticas envolvendo solubilidade, observou-se que: solutos polares dissolvem com maior facilidade em solutos polares, como a água utilizada na prática, enquanto compostos apolares também dissolvem com maior facilidade em solventes apolares, como o hexano. 

A observação contribui para uma melhor compreensão dos princípios da química, no qual, a escolha adequada de solventes e ter uma compreensão sobre condutividade são essenciais para desenvolver produtos e processos.
